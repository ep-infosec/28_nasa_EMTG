\documentclass[11pt]{article}

%% PACKAGES
\usepackage{graphicx}
\usepackage{verbatim}
\usepackage{url}
\usepackage[printonlyused]{acronym}
\usepackage[ruled]{algorithm}
\usepackage{amsmath,amssymb,amsfonts,amsthm}
\usepackage{overpic}
\usepackage{calc}
\usepackage{color}
%\usepackage{times}
%\usepackage{ragged2e}
 \usepackage[margin=1.0in]{geometry}
\usepackage[colorlinks=false]{hyperref}
\usepackage{textcomp}
\usepackage{cite}
\usepackage{mdwlist}
\usepackage{subfiles}
\usepackage{enumitem}
\usepackage{calc}
\usepackage{array}
\usepackage{units}
\usepackage{arydshln,leftidx,mathtools}
\usepackage[caption=false,font=footnotesize]{subfig}
\usepackage{relsize}
\usepackage{float}
\usepackage{makecell}

\usepackage{algorithm}
\usepackage[noend]{algpseudocode}

\makeatletter
\let\@tmp\@xfloat
\usepackage{fixltx2e}
\let\@xfloat\@tmp
\makeatother

\usepackage[subfigure]{tocloft}
\usepackage[singlespacing]{setspace}
%\usepackage[nodisplayskipstretch]{setspace}
%\setstretch{1.0}

%\renewcommand\cftsecafterpnum{\vskip\baselineskip}
%\renewcommand\cftsubsecafterpnum{\vskip\baselineskip}
%\renewcommand\cftsubsubsecafterpnum{\vskip\baselineskip}

%\usepackage{mathtools}
%\usepackage[framed]{mcode}

\usepackage{pgfplots}

\usepackage{cancel}

\usepackage{tikz}
\usetikzlibrary{calc,patterns,decorations.pathmorphing,decorations.markings,fit,backgrounds}

\usepackage[strict]{changepage} %use to manually place figs/tables to get them within the margins

\makeatletter
\g@addto@macro\normalsize{%
  \setlength\abovedisplayskip{0.25pt}
  \setlength\belowdisplayskip{0.25pt}
  \setlength\abovedisplayshortskip{0.25pt}
  \setlength\belowdisplayshortskip{0.25pt}
}
\makeatother



\setlength{\parskip}{\baselineskip}

%% GRAPHICS PATH
\graphicspath{{./pictures/pdf/}{./pictures/ps/}{./pictures/png/}}

%% TODO
\newcommand{\todo}[1]{\vspace{5 mm}\par \noindent \framebox{\begin{minipage}[c]{0.98 \columnwidth} \ttfamily\flushleft \textcolor{red}{#1}\end{minipage}}\vspace{5 mm}\par}

%% MACROS
\providecommand{\abs}[1]{\lvert#1\rvert}
\providecommand{\norm}[1]{\lVert#1\rVert}
\providecommand{\dualnorm}[1]{\norm{#1}_\ast}
\providecommand{\set}[1]{\lbrace\,#1\,\rbrace}
\providecommand{\cset}[2]{\lbrace\,{#1}\nobreak\mid\nobreak{#2}\,\rbrace}
\providecommand{\onevect}{\mathbf{1}}
\providecommand{\zerovect}{\mathbf{0}}
\providecommand{\field}[1]{\mathbb{#1}}
\providecommand{\C}{\field{C}}
\providecommand{\R}{\field{R}}
\providecommand{\polar}{\triangle}
\providecommand{\Cspace}{\mathcal{Q}}
\providecommand{\Fspace}{\mathcal{F}}
\providecommand{\free}{\text{\{}\mathsf{free}\text{\}}}
\providecommand{\iff}{\Leftrightarrow}
\providecommand{\qstart}{q_\text{initial}}
\providecommand{\qgoal}{q_\text{final}}
\providecommand{\contact}[1]{\Cspace_{#1}}
\providecommand{\feasible}[1]{\Fspace_{#1}}
\providecommand{\prob}[2]{p(#1|#2)}
\providecommand{\prior}[1]{p(#1)}
\providecommand{\Prob}[2]{P(#1|#2)}
\providecommand{\Prior}[1]{P(#1)}
\providecommand{\parenth}[1] {\left(#1\right)}
\providecommand{\braces}[1] {\left\{#1\right\}}
\providecommand{\micron}{\hbox{\textmu m}}

%% MATH FUNCTION NAMES
\DeclareMathOperator{\conv}{conv}
\DeclareMathOperator{\cone}{cone}
\DeclareMathOperator{\homog}{homog}
\DeclareMathOperator{\domain}{dom}
\DeclareMathOperator{\range}{range}
\DeclareMathOperator{\argmax}{arg\,max}
\DeclareMathOperator{\argmin}{arg\,min}
\DeclareMathOperator{\area}{area}
\DeclareMathOperator{\sign}{sign}
\DeclareMathOperator{\mathspan}{span}
\DeclareMathOperator{\sn}{sn}
\DeclareMathOperator{\cn}{cn}
\DeclareMathOperator{\dn}{dn}
\DeclareMathOperator*{\minimize}{minimize}

\DeclareMathOperator{\atan2}{atan2}

\newtheorem{theorem}{Theorem}
\newtheorem{lemma}[theorem]{Lemma}

%\setlength{\RaggedRightParindent}{2em}
%\setlength{\RaggedRightRightskip}{0pt plus 3em}
%\pagestyle{empty}




\title{{\Huge EMTGv9 Constraint Scripting}}
\vspace{0.5cm}
\author
{
	Donald H. Ellison \thanks{Aerospace Engineer, NASA Goddard Space Flight Center, Flight Dynamics and Mission Design Branch Code 595},
	Jacob A. Englander \thanks{Aerospace Engineer, NASA Goddard Space Flight Center, Flight Dynamics and Mission Design Branch Code 595}
	Ryo Nakamura \thanks{Aerospace Engineer, NASA Goddard Space Flight Center, Flight Dynamics and Mission Design Branch Code 595, detailed from JAXA}
}
\vspace{0.5cm}

\date{}

\begin{document}

\begin{titlepage}
\maketitle
%\thispagestyle{empty}
\begin{table}[H]
	\centering
	\begin{tabular}{|l|l|l|}
		\hline
		\textbf{Revision Date} & \textbf{Author} & \textbf{Description of Change} \\ \hline
		\date{November 20, 2018} & Donald Ellison & Listing of current implemented constraints \\
		\hline
		\date{January 16, 2019} & Jacob Englander & \makecell[l]{Updated to reflect new organization of .emtgopt file.\\Added overview section.} \\
		\hline
		\date{December 21, 2018} & Jacob Englander & \makecell[l]{Added phase distance constraint, boundary distance constraint,\\ MGAnDSMs maneuver constraints, PSFB thrust angle constraint,\\ and PSFB duty cycle control} \\
		\hline
		\date{January 16, 2018} & Jacob Englander & \makecell[l]{Added introductory material and instructions for how to use the \\scripted constraints with the new modular options file format.} \\
		\hline
		\date{January 23, 2018} & Jacob Englander & \makecell[l]{Added RBP, RPB, RRP, and RPR angle constraint documentation.} \\
		\hline
		\date{February 5, 2019} & Jacob Englander & \makecell[l]{Updated to new options file spec, commentable constraints,\\and BPT angle with constant bounds.} \\
		\hline
		\date{February 11, 2019} & Jacob Englander & \makecell[l]{Added BCF latitude and longitude boundary constraints. Changed\\ parallel shooting constraint text to recognize that PSFB is no longer\\ the only parallel shooting phase type.}\\
		\date{February 14, 2019} & Jacob Englander & \makecell[l]{Added velocity declination constraint.}\\
		\hline
		\date{February 28, 2019} & Jacob Englander & \makecell[l]{Added vertical flight path angle and velocity magnitude constraints.}\\
		\hline
		\date{October 10, 2019} & Ryo Nakamura & \makecell[l]{Added spinning body-relative velocity magnitude constraint.}\\
		\hline
    \date{October 23, 2019} & Ryo Nakamura & \makecell[l]{Added detic latitude, detic altitude, spherical velocity azimuth,\\ relative velocity azimuth, and relative velocity HFPA constraint.}\\
		\hline
	\end{tabular}
\end{table}
\end{titlepage}



\newpage
\tableofcontents
\thispagestyle{empty}
\newpage

\clearpage
\setcounter{page}{1}




\section*{List of Acronyms}
\begin{acronym}
%To define the acronym and include it in the list of acronyms: \acro{acronym}{definition}
%To define the acronym and exclude it from the list of acronyms:  \acro{acronym}{definition}
%
%\ac{acronym} Expand and identify the acronym the first time; use only the acronym thereafter
%\acf{acronym} Use the full name of the acronym.
%\acs{acronym} Use the acronym, even before the first corresponding \ac command
%\acl{acronym}  Expand the acronym without using the acronym itself.
%
%

\acro{ACO}{Ant Colony Optimization}
\acro{AD}{Automatic Differentiation}
\acro{ADL}{Architecture Design Laboratory}
\acro{AES}{Advanced Exploration Systems}
\acro{AGA}{aerogravity assist}
\acro{ALARA}{As Low As Reasonably Achievable}
\acro{API}{application programming interface}
\acro{BB}{branch and bound}
\acro{BVP}{Boundary Value Problem}
\acro{CATO}{Computer Algorithm for Trajectory Optimization}
\acro{CL}{confidence level}
\acro{CONOPS}{concept of operations}
\acro{COV}{Calculus of Variations}
\acro{D/AV}{Descent/Ascent Vehicle}
\acro{DE}{Differential Evolution}
\acro{DLA}{Declination of Launch Asymptote}
\acro{DPTRAJ/ODP}{Double Precision Trajectory and Orbit Determination Program}
\acro{DSH}{Deep Space Habitat}
\acro{DSN}{Deep Space Network}
\acro{DSMPGA}{Dynamic-Size Multiple Population Genetic Algorithm}
\acro{EB}{Evolutionary Branching}
\acro{ECLSS}{environmental control and life support system}
\acro{ELV}{expendable launch vehicle}
\acro{EMME}{Earth to Mars, Mars to Earth}
\acro{EMMVE}{Earth to Mars, Mars to Venus to Earth}
\acro{EMTG}{Evolutionary Mission Trajectory Generator}
\acro{EVMME}{Earth to Venus to Mars, Mars to Earth}
\acro{EVMMVE}{Earth to Venus to Mars, Mars to Venus to Earth}
\acro{ERRV}{Earth Return Re-entry Vehicle}
\acro{FISO}{Future In-Space Operations}
\acro{FMT}{Fast Mars Transfer}
\acro{GASP}{Gravity Assist Space Pruning}
\acro{GCR}{galactic cosmic radiation}
\acro{GRASP}{Greedy Randomized Adaptive Search Procedure}
\acro{GSFC}{Goddard Space Flight Center}
\acro{GTOC}{Global Trajectory Optimization Competition}
\acro{GTOP}{Global Trajectory Optimization Problem}
\acro{HAT}{Human Architecture Team}
\acro{HGGA}{Hidden Genes Genetic Algorithm}
\acro{IMLEO}{Initial Mass in \acl{LEO}}
\acro{IPOPT}{Interior Point OPTimizer}
\acro{ISS}{International Space Station}
\acro{JHUAPL}{Johns Hopkins University Applied Physics Laboratory}
\acro{JSC}{Johnson Space Center}
\acro{KKT}{Karush-Kuhn-Tucker}
\acro{LEO}{Low Earth Orbit}
\acro{LRTS}{lazy race tree search}
\acro{MONTE}{Mission analysis, Operations, and Navigation Toolkit Environment}
\acro{MCTS}{Monte Carlo tree search}
\acro{MGA}{Multiple Gravity Assist}
\acro{MIRAGE}{Multiple Interferometric Ranging Analysis using GPS Ensemble}
\acro{MOGA}{Multi-Objective Genetic Algorithm}
\acro{MOSES}{Multiple Orbit Satellite Encounter Software}
\acro{MPI}{message passing interface}
\acro{MPLM}{Multi-Purpose Logistics Module}
\acro{MSFC}{Marshall Space Flight Center}
\acro{NELLS}{NASA Exhaustive Lambert Lattice Search}
\acro{NSGA}{Non-Dominated Sorting Genetic Algorithm}
\acro{NSGA-II}{Non-Dominated Sorting Genetic Algorithm II}
\acro{NHATS}{Near-Earth Object Human Space Flight Accessible Targets Study}
\acro{NTP}{Nuclear Thermal Propulsion}
\acro{OD}{orbit determination}
\acro{OOS}{On-Orbit Staging}
\acro{PCC}{Pork Chop Contour}
\acro{PEL}{permissible exposure limits}
\acro{PLATO}{PLAnetary Trajectory Optimization}
\acro{REID}{risk of exposure-induced death}
\acro{RTBP}{Restricted Three Body Problem}
\acro{SA}{Simulated Annealing}
\acro{SLS}{Space Launch System}
\acro{SNOPT}{Sparse Nonlinear OPTimizer}
\acro{SOI}{sphere of influence}
\acro{SPE}{solar particle events}
\acro{SQP}{sequential quadratic programming}
\acro{SRAG}{Space Radiation Analysis Group}
\acro{TEI}{Trans-Earth Injection}
\acro{TOF}{time of flight}
\acro{TPBVP}{Two Point Boundary Value Problem}
\acro{TMI}{Trans-Mars Injection}
\acro{VARITOP}{Variational calculus Trajectory Optimization Program}
\acro{VILM}{v-infinity leveraging maneuver}
\acro{MOI}{Mar Orbit Injection}
\acro{PCM}{Pressurized Cargo Module}
\acro{STS}{Space Transportation System}
\acro{EDS}{Earth Departure Stage}
\acro{NEO}{near-Earth asteroid}
\acro{IDC}{Integrated Design Center}
\acro{SEP}{solar-electric propulsion}
\acro{SRP}{solar radiation pressure}
\acro{NEP}{nuclear-electric propulsion}
\acro{REP}{radioisotope-electric propulsion}
\acro{DRM}{Design Reference Missions}

\acro{ASCII}{American Standard Code for Information Interchange}
\acro{AU}{Astronomical Unit}
\acro{BWG}{Beam Waveguides}
\acro{CCB}{Configuration Control Board}
\acro{CMO}{Configuration Management Office}
\acro{CODATA}{Committee on Data for Science and Technology}
\acro{DEEVE}{Dynamically Equivalent Equal Volume Ellipsoid}
\acro{DRA}{Design Reference Asteroid}
\acro{EME2000}{Earth Centered, Earth Mean Equator and Equinox of J2000 (Coordinate Frame)}
\acro{EOP}{Earth Orientation Parameters}
\acro{ET}{Ephemeris Time}
\acro{FDS}{Flight Dynamics System}
\acro{FTP}{File Transfer Protocol}
\acro{GSFC}{Goddard Space Flight Center}
\acro{PI}{Principal Investigator}
\acro{HEF}{High Efficiency}
\acro{IAG}{International Association of Geodesy}
\acro{IAU}{International Astronomical Union}
\acro{IERS}{International Earth Rotation and Reference Systems Service}
\acro{ICRF}{International Celestial Reference Frame}
\acro{ITRF}{International Terrestrial Reference System}
\acro{IOM}{Interoffice Memorandum}
\acro{JD}{Julian Date}
\acro{JPL}{Jet Propulsion Laboratory}
\acro{LM}{Lockheed Martin}
%\acro{LP150Q}{}
%\acros{LP100K}{}
\acro{MAVEN}{Mars Atmosphere and Volatile EvolutioN}
\acro{MJD}{Modified Julian Date}
\acro{MOID}{Minimum Orbit Intersection Distance}
\acro{MPC}{Minor Planet Center}
\acro{NASA}{National Aeronautics and Space Administration}
\acro{NDOSL}{\ac{NASA} Directory of Station Locations}
\acro{NEA}{near-Earth asteroid}
\acro{NEO}{near-Earth object}
\acro{NIO}{Nav IO}
\acro{OSIRIS-REx}{Origins Spectral Interpretation Resource Identification Security-Regolith Explorer}
\acro{PHA}{Potentially Hazardous Asteroid}
\acro{PHO}{Potentially Hazardous Object}
\acro{SBDB}{Small-Body Database}
\acro{SI}{International System of Units}
\acro{SPICE}{Spacecraft Planet Instrument Camera-matrix Events}
\acro{SPK}{SPICE Kernel}
\acro{SRC}{Sample Return Capsule}
\acro{SSD}{Solar System Dynamics}
\acro{STK}{Systems Tool Kit}
\acro{TAI}{International Atomic Time}
\acro{TBD}{To Be Determined}
\acro{TBR}{To Be Reviewed}
\acro{TCB}{Barycentric Coordinate Time}
\acro{TDB}{Temps Dynamiques Barycentrique, Barycentric Dynamical Time}
\acro{TDT}{Terrestrial Dynamical Time}
\acro{TT}{Terrestrial Time}
\acro{URL}{Uniform Resource Locator}
\acro{UT}{Universal Time}
\acro{UT1}{Universal Time Corrected for Polar Motion}
\acro{UTC}{Coordinated Universal Time}
\acro{USNO}{U. S. Naval Observatory}
\acro{YORP}{Yarkovsky-O'Keefe-Radzievskii-Paddack}

\acro{NLP}{nonlinear program}
\acro{MBH}{monotonic basin hopping}
\acro{MBH-C}{monotonic basin hopping with Cauchy hops}
\acro{FBS}{forward-backward shooting}
\acro{MGALT}{multiple gravity assist with low-thrust}
\acro{MGALTS}{multiple gravity assist with low-thrust using a Sundman transformation}
\acro{MGA-1DSM}{Multiple Gravity Assist with One Deep Space Maneuver}
\acro{MGAnDSMs}{multiple gravity assist with n deep-space maneuvers using shooting}
\acro{PSFB}{Parallel Shooting with Finite-Burn}
\acro{PSBI}{Parallel Shooting with Bounded Impulses}
\acro{FBLT}{finite-burn low-thrust}
\acro{ESA}{European Space Agency}
\acro{ACT}{Advanced Concepts Team}
\acro{IRAD}{independent research and development}
%\acro{Isp}[$\text{I}_{sp}$]{specific impulse}
\acro{GA}{genetic algorithm}
\acro{GALLOP}{ Gravity Assisted Low-thrust Local Optimization Program}
\acro{MALTO}{Mission Analysis Low-Thrust Optimization}
\acro{PaGMO}{Parallel Global Multiobjective Optimizer}
\acro{FRA}{feasible region analysis}
\acro{CP}{conditional penalty}
\acro{HOC}{hybrid optimal control}
\acro{HOCP}{hybrid optimal control problem}
\acro{PSO}{particle swarm optimization}
\acro{SEPTOP}{Solar Electric Propulsion Trajectory Optimization Program}
\acro{STOUR}{Satellite Tour Design Program}
\acro{STOUR-LTGA}{Satellite Tour Design Program - Low Thrust, Gravity Assist}
\acro{PaGMO}{Parallel Global Multiobjective Optimizer}
\acro{SDC}{static/dynamic control}
\acro{DDP}{Differential Dynamic Programming}
\acro{HDDP}{Hybrid Differential Dynamic Programming}
\acro{ACT}{Advanced Concepts Team}
\acro{GMAT}{General Mission Analysis Toolkit}
\acro{BOL}{beginning of life}
\acro{EOL}{end of life}
\acro{KSC}{Kennedy Space Center}
\acro{VSI}{variable \ac{Isp}}
\acro{RTG}{radioisotope thermal generator}
\acro{ASRG}{advanced Stirling radiosotope generator}
\acro{ARRM}{Asteroid Robotic Redirect Mission}
\acro{AATS}{Alternative Architecture Trade Study}
\acro{PPU}{power processing unit}
\acro{STM}{state transition matrix}
\acro{MTM}{maneuver transition matrix}
\acro{BCI}{body-centered inertial}
\acro{BCF}{body-centered fixed}
\acro{UTTR}{Utah Test and Training Range}
\acro{EPV}{equatorial projection of $\mathbf{v}_\infty$}
\acro{KBO}{Kuiper belt object}
\acro{DSM}{deep-space maneuver}
\acro{BPT}{body-probe-thrust}
\acro{4PL}{four parameter logistic}
\acro{BCF}{body-centered fixed}

\end{acronym}

% --------------------------------------------------------------------------------------------------------------------------
% --------------------------------------------------------------------------------------------------------------------------


\section{Overview}
\label{sec:overview}

EMTGv9 allows the user to specify a wide range of scripted constraints beyond what is available in the GUI. This document describes the three classes of constraints - boundary constraints, maneuver constraints, and phase distance constraints (also known as path constraints).

EMTGv9's .emtgopt input file contains a separate, copyable block for each journey in the mission. At the end of each journey, blocks exist for each of the three classes of constraints as shown below:

\begin{verbatim}
#Maneuver constraint code
#Works for absolute and relative epochs and also magnitudes
BEGIN_MANEUVER_CONSTRAINT_BLOCK
END_MANEUVER_CONSTRAINT_BLOCK


#Boundary constraint code
BEGIN_BOUNDARY_CONSTRAINT_BLOCK
END_BOUNDARY_CONSTRAINT_BLOCK


#Phase distance constraint code
BEGIN_PHASE_DISTANCE_CONSTRAINT_BLOCK
END_PHASE_DISTANCE_CONSTRAINT_BLOCK
\end{verbatim}

Constraints are always prefixed with a tag representing the phase that they constrain. Tags are indexed from zero. For example, constraints on the first phase of the journey are prefixed \texttt{p0}, and constraints on the fifth phase are prefixed \texttt{p4}. The user can also place constraints on the \textit{last} phase of the journey using the prefix \texttt{pEnd}.

Internally, the constraints are stored in the JourneyOptions structure as lists of strings:

\begin{verbatim}
JourneyOptions.ManeuverConstraintDefinitions
JourneyOptions.BoundaryConstraintDefinitions
JourneyOptions.PhaseDistanceConstraintDefinitions
\end{verbatim}

PyEMTG collects the journey-level lists of constraints into ``master'' constraint lists that exist at the mission level as shown below. To do this, call \texttt{MissionOptions.AssembleMasterConstraintVectors}. You can also disassemble the master constraint list back to the journey level via 

 \texttt{MissionOptions.DisassembleMasterDecisionVector}.

\begin{verbatim}
MissionOptions.ManeuverConstraintDefinitions
MissionOptions.BoundaryConstraintDefinitions
MissionOptions.PhaseDistanceConstraintDefinitions
\end{verbatim}

If you wish to edit scripted constraints at the journey level via Python, do the following:

\begin{verbatim}
MissionOptions.DisassembleMasterDecisionVector()
<your constraint edits here>
MissionOptions.AssembleMasterConstraintVectors()
\end{verbatim}

Alternatively, if you wish to edit at the mission level, perform the procedure below. However, you will need to prefix \texttt{jJ} to the constraint names when manipulating the master constraint definition lists, where \texttt{J} is the index of the journey that you wish to edit.

\begin{verbatim}
MissionOptions.AssembleMasterConstraintVectors()
<your constraint edits here>
MissionOptions.DisassembleMasterDecisionVector()
\end{verbatim}

Remember to call \texttt{MissionOptions.DisassembleMasterDecisionVector} and \\ \texttt{MissionOptions.AssembleMasterDecisionVector} as appropriate so that the mission and journey-level constraint definition lists remain in sync. In a future update, we will find a way to make this happen automatically.

The user may comment out a constraint with the \texttt{\#} character. The constraint will remain in the script through ingest and rewrite in PyEMTG and EMTG, but will not be used in optimization.

\section{Boundary Constraints}
\label{sec:boundaryConstraints}

Boundary constraints may be applied to any boundary condition class. In order to apply the constraint to the left boundary of a particular phase/journey, the constraint should be tagged as a departure constraint (e.g. p2\_departure\_orbitperiod\_1yr\_1.1yr). Likewise, for a right boundary, it would be tagged as ``arrival''.

Boundary constraints are specified in the boundary constraint block at the end of each journey in the .emtgopt file.

\subsection{Classical Orbit Element Boundary Constraints}
\label{sec:boundaryCOEconstraints}

Distance-based constraints may be specified in kilometers (km), nautical miles (nmi), \ac{AU} (AU) or universe length units (LU). Angular constraints may be specified in degrees (deg) or radians (rad). Time constraints may be specified in seconds (sec), hours (hr), days (d) or years (yr). The orientation COEs require that frame specification (e.g. ICRF, J2000BCI etc.).

\subsubsection{Semimajor Axis}
\label{sec:boundarySMAconstraint}

\begin{verbatim}
BEGIN_BOUNDARY_CONSTRAINT_BLOCK
pP_arrival_SMA_10000km_12000km
END_BOUNDARY_CONSTRAINT_BLOCK
\end{verbatim}

\subsubsection{Inclination}
\label{sec:boundaryINCconstraint}
\begin{verbatim}
BEGIN_BOUNDARY_CONSTRAINT_BLOCK
pP_departure_INC_40deg_40.1deg_J2000BCI
END_BOUNDARY_CONSTRAINT_BLOCK
\end{verbatim}

\subsubsection{Eccentricity}
\label{sec:boundaryECCconstraint}

\begin{verbatim}
BEGIN_BOUNDARY_CONSTRAINT_BLOCK
pP_arrival_ECC_0.6_0.7
END_BOUNDARY_CONSTRAINT_BLOCK
\end{verbatim}

\subsubsection{Right Ascension of the Ascending Node}
\label{sec:boundaryRAANconstraint}

\begin{verbatim}
BEGIN_BOUNDARY_CONSTRAINT_BLOCK
pP_arrival_RAAN_0.1rad_0.15rad_ICRF
END_BOUNDARY_CONSTRAINT_BLOCK
\end{verbatim}

\subsubsection{Argument of Periapse}
\label{sec:boundaryAOPconstraint}

\begin{verbatim}
BEGIN_BOUNDARY_CONSTRAINT_BLOCK
pP_arrival_AOP_0.1rad_0.15rad_TrueOfDateBCF
END_BOUNDARY_CONSTRAINT_BLOCK
\end{verbatim}

\subsubsection{True Anomaly}
\label{sec:boundaryTAconstraint}

\begin{verbatim}
BEGIN_BOUNDARY_CONSTRAINT_BLOCK
pP_arrival_trueanomaly_10deg_15deg_ICRF
END_BOUNDARY_CONSTRAINT_BLOCK
\end{verbatim}

\subsection{COE Derived Constraints}
\label{sec:boundaryCOEderivedConstraints}

\subsubsection{Orbit Period}
\label{sec:boundaryOrbitPeriod}

For closed orbits (i.e. $a > 0.0$) the spacecraft's orbital period $T$ can be constrained:

\begin{equation}
	T = \sqrt{\frac{a^3}{\mu}}
\end{equation}

\begin{verbatim}
	BEGIN_BOUNDARY_CONSTRAINT_BLOCK
	pP_departure_orbitperiod_5.0d_5.3d
	END_BOUNDARY_CONSTRAINT_BLOCK
\end{verbatim}

In general, it is preferable to specify this constraint as an SMA constraint \ref{sec:boundarySMAconstraint} in the event that the spacecraft state results in $a <= 0.0$ (which can happen during optimizer iteration).

\subsubsection{Periapse Distance}
\label{sec:boundaryOrbitPeriapse}

\begin{verbatim}
BEGIN_BOUNDARY_CONSTRAINT_BLOCK
pP_arrival_periapsedistance_10000nmi_12000nmi
END_BOUNDARY_CONSTRAINT_BLOCK
\end{verbatim}

\subsubsection{Apoapse Distance}
\label{sec:boundaryOrbitApoapse}

\begin{verbatim}
BEGIN_BOUNDARY_CONSTRAINT_BLOCK
pP_arrival_apoapsedistance_10000nmi_12000nmi
END_BOUNDARY_CONSTRAINT_BLOCK
\end{verbatim}

\subsection{Orbital Energy}
\label{sec:orbitalEnergyConstraint}

Specific orbital energy $\epsilon$ can be constrained:

\begin{equation}
	\label{eq:orbital_energy}
	\epsilon = \frac{1}{2}v^2 - \frac{\mu}{r}
\end{equation}

\noindent Currently, orbital energy can only be specified in units of $km^2/s^2$.

\begin{verbatim}
	BEGIN_BOUNDARY_CONSTRAINT_BLOCK
	pP_arrival_orbitalenergy_10.0km2s2_10.3km2s2
	END_BOUNDARY_CONSTRAINT_BLOCK
\end{verbatim}

\subsection{Distance Constraint}
\label{sec:boundaryDistanceConstraint}

The user may constrain the distance between any boundary condition and any body in the Universe file. The user specifies the phase prefix, ``arrival'' or ``departure,'' the body of interest as either ``cb'' or an integer index from the Universe file, and lower and upper bounds. The constraint may be posed in km, AU, or Universe LU.

\begin{verbatim}
BEGIN_BOUNDARY_CONSTRAINT_BLOCK
p0_arrival_distanceconstraint_3_0.0au_1.0au
p0_departure_distanceconstraint_cb_0.0au_3.5au
END_BOUNDARY_CONSTRAINT_BLOCK
\end{verbatim}

\subsection{Angle Constraints}
\label{subsec:boundaryAngleConstraint}

The user may constrain the angle between the sun, the spacecraft, and a reference body. This may be done in terms of the Sun-Body-Probe (SBP) or Sun-Probe-Body (SPB). This set of constraints can be used to represent anything from the phase angle at encounter with an asteroid (Sun-asteroid-Probe) or the Sun-Earth-Probe (SEP) or Sun-Probe-Earth (SPE) angles.

There are four versions of these constraints as detailed below. The first two versions apply only to ephemeris-pegged boundary conditions that encode a $v_\infty$ vector, such as intercept/flyby, chemical rendezvous, and orbit insertion. The last two may be applied to any class of boundary condition.

\begin{enumerate}
	\item \textbf{Ephemeris-pegged Reference-Body-Probe (RBP)}. The user only needs to provide the reference body. The constraint uses the boundary position vector to represent the Body and the sum of the boundary position vector and the spacecraft $v_\infty$ vector to represent the Probe. An example would be Sun-asteroid-Probe angle for an ephemeris-pegged asteroid flyby, where the Sun is the Reference and the asteroid is the Body.
	\item \textbf{Ephemeris-pegged Reference-Probe-Body (RPB)}. The user only needs to provide the reference body. The constraint uses the boundary position vector to represent the Body and the sum of the boundary position vector and the spacecraft $v_\infty$ vector to represent the Probe. An example would be Sun-Probe-asteroid angle for an ephemeris-pegged asteroid flyby, where the Sun is the Reference and the asteroid is the Body.
	\item \textbf{Reference-Reference-Probe (RRP)}. The user specifies two reference bodies and the spacecraft's position vector is used for the Probe. One example is Sun-Earth-Probe angle, where the Sun and the Earth are the two references. Another example is Sun-asteroid-Probe angle for a flyby that is specified with any boundary class other than ephemeris-pegged.
	\item \textbf{Reference-Probe-Reference (RPR)}. The user specifies two reference bodies and the spacecraft's position vector is used for the Probe. One example is Sun-Probe-Earth angle, where the Sun and the Earth are the two references. Another example is Sun-Probe-asteroid angle for a flyby that is specified with any boundary class other than ephemeris-pegged.
\end{enumerate}

\subsubsection{RBP and RPB angle constraints}
\label{subsubsec:RBP_and_RPB}

The RBP and RPB constraints are specified as shown below. The user may identify the reference body using either ``cb'' for the central body or an integer encoding a body's index in that Journey's Universe file. The lower and upper bounds on the constraint are specified in degrees.

The RBP and RPB constraints may only be specified for ephemeris-pegged arrival events that encode a $v_\infty$ vector, \textit{i.e.} intercept/flyby, chemical rendezvous, and orbit insertion. Note that because RBP and RPB only work with ephemeris-pegged boundary event, we can make the assumption that the ``body'' is the body associated with the boundary event and so we only need to encode the ``reference'' in the constraint description as shown below.

\begin{verbatim}
BEGIN_BOUNDARY_CONSTRAINT_BLOCK
p0_arrival_RBP_cb_0.0_15.0
p3_arrival_RPB_5_8.3_17.2
END_BOUNDARY_CONSTRAINT_BLOCK
\end{verbatim}


\subsubsection{RRP and RPR angle constraints}
\label{subsubsec:RRP_and_RP}

The RRP and RPR constraints are specified as shown below. The syntax for RRP and RPR requires that the user specify two reference bodies. The user may identify the reference body using either ``cb'' for the central body or an integer encoding a body's index in that Journey's Universe file. The lower and upper bounds on the constraint are specified in degrees.

The RRP and RPR constraints may be applied to any boundary event.

\begin{verbatim}
BEGIN_BOUNDARY_CONSTRAINT_BLOCK
p0_arrival_RRP_cb_3_0.0_15.0
p3_departure_RPR_5_7_8.3_17.2
END_BOUNDARY_CONSTRAINT_BLOCK
\end{verbatim}

\subsection{Angular Momentum Reference Angle Constraint}
\label{subsec:angular_momentum_reference_angle}

The user may choose to constrain the angle between the spacecraft's ICRF angular momentum vector relative to the central body and the vector to some reference body. For example, the user may need to constrain that the final orbit of the spacecraft about some body be in the terminator plane, \textit{i.e.} the angle between the vector to the sun and the angular momentum vector is zero.

The user may supply this constraint in degrees or radians.

\begin{verbatim}
BEGIN_BOUNDARY_CONSTRAINT_BLOCK
p0_arrival_angularMomentumReferenceAngle_cb_0.0deg_15.0deg
END_BOUNDARY_CONSTRAINT_BLOCK
\end{verbatim}


\subsection{Departure or Arrival Maneuver Thruster Set}
\label{subsec:boundary_constrainThrusterSet}

The user may direct a particular chemical departure or arrival maneuver to be either ``monoprop'' or ``biprop.'' EMTG will then use the $I_{sp}$ specified for monoprop or biprop in the spacecraft file.

This class of constraint is applied to departure type \texttt{0: launch or direct insertion} (when a launch vehicle is not used) and arrival types \texttt{0: insertion into parking orbit (use chemical Isp)} and \texttt{1: rendezvous (with chemical maneuver)}, as well as powered patched-conic flybys. All other boundary types will ignore this class of constraint.

\begin{verbatim}
BEGIN_BOUNDARY_CONSTRAINT_BLOCK
p0_arrival_monoprop
p3_departure_biprop
END_BOUNDARY_CONSTRAINT_BLOCK
\end{verbatim}

\subsection{Longitude}
\label{subsec:BCF_longitude}

The user may constrain longitude in ICRF, J2000BCF or TrueOfDateBCF. Longitude is computed as,

\begin{equation}
	\label{eq:longitude}
	\lambda = \atan2\left (y, x \right)
\end{equation}

\noindent and is given in the range $\left[-180.0, 180.0 \right]$. The user provides lower and upper bounds on longitude, in degrees.


\begin{verbatim}
BEGIN_BOUNDARY_CONSTRAINT_BLOCK
p0_departure_longitude_84.0_85.0
END_BOUNDARY_CONSTRAINT_BLOCK
\end{verbatim}

\subsection{BCF Latitude}
\label{subsec:BCF_latitude}

The user may constrain latitude in the J2000 \ac{BCF} frame as defined in the current journey's Universe file. This is also known as geocentric latitude as opposed to geodetic latitude. Latitude is computed as,

\begin{equation}
	\label{eq:latitude}
	\begin{aligned}
		\theta = \atan2\left (z_{J2000BCF}, r_{xy_{J2000BCF}} \right)\\
		r_{xy_{J2000BCF}} = \sqrt{\left(x_{J2000BCF}^2 + y_{J2000BCF}^2\right)}
	\end{aligned}
\end{equation}

\noindent and is given in the range $\left[-180.0, 180.0 \right]$. The user provides lower and upper bounds on longitude, in degrees.

Note that this constraint is computed in the \textit{universe's central body's} J2000 \ac{BCF} frame. This is true regardless of boundary class.

\begin{verbatim}
BEGIN_BOUNDARY_CONSTRAINT_BLOCK
p0_departure_BCFlatitude_-29.0_-28.0
END_BOUNDARY_CONSTRAINT_BLOCK
\end{verbatim}

\subsection{(Geo)Detic Latitude}
\label{subsec:detic_latitude}

The user may constrain (geo)detic latitude. The user may specify this constraint in J2000BCF and TrueOfDateBCI frames. The user provides lower and upper bounds on latitude, in degrees.
Note that this constraint has a singularity at the pole.

\begin{verbatim}
BEGIN_BOUNDARY_CONSTRAINT_BLOCK
p0_arrival_DeticLatitude_25.0_26.0_trueofdatebcf
END_BOUNDARY_CONSTRAINT_BLOCK
\end{verbatim}

\subsection{(Geo)Detic Altitude}
\label{subsec:detic_altitude}

The user may constrain (geo)detic altitude. The user may specify this constraint in J2000BCF and TrueOfDateBCI frames. The user provides lower and upper bounds on altitude, in km.
Note that (derivative of) this constraint has a singularity at the pole.

\begin{verbatim}
BEGIN_BOUNDARY_CONSTRAINT_BLOCK
p0_arrival_DeticAltitude_206.0_207.0_trueofdatebcf
END_BOUNDARY_CONSTRAINT_BLOCK
\end{verbatim}

\subsection{(Geo)Detic Target Body Elevation}
\label{subsec:detic_altitude}

The user may constrain the (geo)detic elevation of a target body with respect to the surface normal vector of a spheroid with flattening factor $f$. The target body is specified using an integer encoding a body's index in that Journey's Universe file. The user may specify this constraint in J2000BCF and TrueOfDateBCI frames. The user provides lower and upper bounds on detic elevation, in degrees.

%\noindent Note that (derivative of) this constraint has a singularity at the pole.

\begin{verbatim}
BEGIN_BOUNDARY_CONSTRAINT_BLOCK
p0_arrival_DeticElevation_10_70.0_90.0_trueofdatebcf
END_BOUNDARY_CONSTRAINT_BLOCK
\end{verbatim}

\subsection{Velocity Declination}
\label{subsec:velocity_declination}

The user may constrain velocity declination, given by:

\begin{equation}
	\delta _v = \atan2 \left(v_z, v_{xy} \right)
\end{equation}

\noindent where $v_{xy}$ is the projection of the velocity vector onto the x-y plane. The user may specify this constraint in ICRF, J2000BCI, J2000BCF, TrueOfDateBCI, and TrueOfDateBCF frames.

\begin{verbatim}
BEGIN_BOUNDARY_CONSTRAINT_BLOCK
p0_departure_VelocityDeclination_-20.0_5.0_ICRF
END_BOUNDARY_CONSTRAINT_BLOCK
\end{verbatim}

\subsection{Vertical Flight Path Angle (VFPA)}
\label{subsec:velocity_vertical_flight_path_angle}
The user may constrain vertical flight path angle, given by:

\begin{equation}
\psi = \arccos \left(\frac{\mathbf{r} \cdot \mathbf{v} }{rv}\right)
\end{equation}

\noindent where the $CF$ denotes the constraint frame. The user may specify this constraint in ICRF, J2000BCI, J2000BCF, TrueOfDateBCI, and TrueOfDateBCF frames.

\begin{verbatim}
BEGIN_BOUNDARY_CONSTRAINT_BLOCK
p0_arrival_VFPA_97.0_98.0_J2000BCF
END_BOUNDARY_CONSTRAINT_BLOCK
\end{verbatim}

\subsection{Velocity magnitude}
\label{subsec:velocity_magnitude}
The user may constrain velocity magnitude in km/s. The user may specify this constraint in ICRF, J2000BCI, J2000BCF, TrueOfDateBCI, and TrueOfDateBCF frames.

\begin{verbatim}
BEGIN_BOUNDARY_CONSTRAINT_BLOCK
p0_arrival_VelocityMagnitude_11.9_12.1_J2000BCF
END_BOUNDARY_CONSTRAINT_BLOCK
\end{verbatim}

\subsection{Velocity Spherical Azimuth}
\label{subsec:velocity_sphericalazimuth}
The user may constrain spherical azimuth angle of inertial velocity vector in degrees. The spherical azimuth angle is given as
\begin{equation}
	\label{eq:velocity_Az}
	\begin{aligned}
		&Az = \arctan{\frac{v_{east}}{v_{north}}}
	\end{aligned}
\end{equation}
,where $v_{east}$ and $v_{north}$ are inertial velocity elements at the spherical local coordinate as
\begin{equation}
	\label{eq:velocity_local}
	\begin{aligned}
		\begin{bmatrix}
		v_{up}\\
		v_{east}\\
		v_{north}
		\end{bmatrix} =
		\begin{bmatrix}
		\cos{\phi} & 0 & \sin{\phi}\\
		0 & 1 & 0\\
		-\sin{\phi} & 0 & \cos{\phi}
		\end{bmatrix}
		\begin{bmatrix}
		\cos{\lambda} & \sin{\lambda} & 0\\
		-\sin{\lambda}& \cos{\lambda} & 0\\
		0 & 0 & 1
		\end{bmatrix}
		\begin{bmatrix}
		v_{r}\\
		v_{r}\\
		v_{r}
		\end{bmatrix}
	\end{aligned}
\end{equation}
Here, $\phi$ is a (geo)centric latitude and $\lambda$ is a longitude. The user may specify this constraint in J2000BCF and TrueOfDateBCF frames.

\begin{verbatim}
BEGIN_BOUNDARY_CONSTRAINT_BLOCK
p0_arrival_VelocitySphericalAzimuth_50.1_50.2_trueofdatebcf
END_BOUNDARY_CONSTRAINT_BLOCK
\end{verbatim}

\subsection{Relative Velocity magnitude}
\label{subsec:relativevelocity_magnitude}
The user may constrain magnitude of velocity vector relative to central body's ground (or atmosphere) in km/s. The relative velocity vector and the magnitude are given by
\begin{equation}
	\label{eq:rel_velocity}
	\begin{aligned}
		\vec{v_r} = \vec{v} - \vec{\omega} \times \vec{r} = \vec{v} -
		\begin{bmatrix}
		0 & -\omega_3 & \omega_2 \\
		\omega_3 & 0 & -\omega_1 \\
		-\omega_2 & \omega_1 & 0
		\end{bmatrix}
		\vec{r}
	\end{aligned}
\end{equation}
\begin{equation}
	\label{eq:rel_velocity_magnitude}
	\begin{aligned}
		&v_r = \left|\vec{v_r}\right| = \sqrt{v_{rx}^{2}+v_{ry}^{2}+v_{rz}^{2}}
	\end{aligned}
\end{equation}
The user may specify this constraint in J2000BCF and TrueOfDateBCF frames.

\begin{verbatim}
BEGIN_BOUNDARY_CONSTRAINT_BLOCK
p0_arrival_RelativeVMagnitude_11.9_12.1_J2000BCF
END_BOUNDARY_CONSTRAINT_BLOCK
\end{verbatim}

\subsection{Relative Velocity Azimuth}
\label{subsec:relativevelocity_azimuth}
The user may constrain azimuth angle of velocity vector relative to central body's ground (or atmosphere) in degrees. The azimuth angle is given as
\begin{equation}
	\label{eq:rel_velocity_Az}
	\begin{aligned}
		&Az = \arctan{\frac{v_{r_{east}}}{v_{r_{north}}}}
	\end{aligned}
\end{equation}
,where $v_{r_{east}}$ and $v_{r_{north}}$ are relative velocity elements at the horizontal local coordinate as
\begin{equation}
	\label{eq:rel_velocity_local}
	\begin{aligned}
		\begin{bmatrix}
		v_{r_{up}}\\
		v_{r_{east}}\\
		v_{r_{north}}
		\end{bmatrix} =
		\begin{bmatrix}
		\cos{\phi} & 0 & \sin{\phi}\\
		0 & 1 & 0\\
		-\sin{\phi} & 0 & \cos{\phi}
		\end{bmatrix}
		\begin{bmatrix}
		\cos{\lambda} & \sin{\lambda} & 0\\
		-\sin{\lambda}& \cos{\lambda} & 0\\
		0 & 0 & 1
		\end{bmatrix}
		\begin{bmatrix}
		v_{rx}\\
		v_{ry}\\
		v_{rz}
		\end{bmatrix}
	\end{aligned}
\end{equation}
Here, $\phi$ is a (geo)detic latitude and $\lambda$ is a longitude. The user may specify this constraint in J2000BCF and TrueOfDateBCF frames.

\begin{verbatim}
BEGIN_BOUNDARY_CONSTRAINT_BLOCK
p0_arrival_RelativeVAzimuth_50.1_50.2_trueofdatebcf
END_BOUNDARY_CONSTRAINT_BLOCK
\end{verbatim}

\subsection{Relative Velocity Horizontal Flight Path Angle(HFPA)}
\label{subsec:relativevelocity_hfpa}
The user may constrain horizontal flight path angle of velocity vector relative to central body's ground (or atmosphere) in degrees. The HFPA is given as
\begin{equation}
	\label{eq:rel_velocity_HFPA}
	\begin{aligned}
		&HFPA = \arctan{\frac{v_{r_{up}}}{\sqrt{v_{r_{north}}^2 + v_{r_{east}}^2}}}
	\end{aligned}
\end{equation}
,where $v_{r_{east}}$ and $v_{r_{north}}$ are relative velocity elements at the horizontal local coordinate as
\begin{equation}
	\label{eq:rel_velocity_local}
	\begin{aligned}
		\begin{bmatrix}
		v_{r_{up}}\\
		v_{r_{east}}\\
		v_{r_{north}}
		\end{bmatrix} =
		\begin{bmatrix}
		\cos{\phi} & 0 & \sin{\phi}\\
		0 & 1 & 0\\
		-\sin{\phi} & 0 & \cos{\phi}
		\end{bmatrix}
		\begin{bmatrix}
		\cos{\lambda} & \sin{\lambda} & 0\\
		-\sin{\lambda}& \cos{\lambda} & 0\\
		0 & 0 & 1
		\end{bmatrix}
		\begin{bmatrix}
		v_{rx}\\
		v_{ry}\\
		v_{rz}
		\end{bmatrix}
	\end{aligned}
\end{equation}
Here, $\phi$ is a (geo)detic latitude and $\lambda$ is a longitude. The user may specify this constraint in J2000BCF and TrueOfDateBCF frames.

\begin{verbatim}
BEGIN_BOUNDARY_CONSTRAINT_BLOCK
p0_arrival_RelativeVHFPA_-14.1_-14.0_trueofdatebcf
END_BOUNDARY_CONSTRAINT_BLOCK
\end{verbatim}


\section{MGAnDSMs Maneuver Constraints}
\label{sec:MGAnDSMs_Maneuver_Constraints}

The \ac{MGAnDSMs} transcription allows the optimizer to place up to $n$ deep-space maneuvers in each phase of a multiple gravity-assist mission. Nominally, the numerical optimizer will place the maneuvers in locally optimal locations based on the \ac{NLP} necessary conditions for optimality. There are, however, many instances where the mission designer will want to constrain the parameters of a particular maneuver for practical reasons including, but not limited to:

\begin{enumerate}
	\item Navigation restrictions: no maneuver pre/post another mission critical event
	\item Engine type: restrict a certain maneuver to be executed with a particular propulsion system (bipropellant or monopropellant thruster sets)
	\item Magnitude: constrain the size of a particular maneuver due to hardware limitations or practical navigation concerns
\end{enumerate}

Individual maneuver constraints are specified in the .emtgopt file. Each journey has a maneuver constraint block:

\begin{verbatim}
BEGIN_MANEUVER_CONSTRAINT_BLOCK
constraint_1
constraint_2
constraint_3
.
.
.
constraint_n
END_MANEUVER_CONSTRAINT_BLOCK
\end{verbatim}

\noindent The basic maneuver constraint syntax identifies the journey and phase that the maneuver is in, as well as its index within the phase (indexed from zero).

\noindent Instructions on how to specify different types of maneuver constraints are provided in the following sections along with examples.

\subsection{Maneuver Epoch}
\label{subsec:MGAnDSMs_constraintEpoch}

Maneuver positions can be specified via their epoch in a variety of ways: an absolute epoch, epoch offset with respect to another mission event (e.g. another maneuver's epoch), or an epoch offset w.r.t. a right or left phase boundary.

\subsubsection{Absolute Epoch}

A maneuver may be fixed in time by specifying either a \ac{MJD} or \ac{JD}.

\begin{verbatim}
BEGIN_MANEUVER_CONSTRAINT_BLOCK
p0b0_epoch_abs_51544.0_51544.5
END_MANEUVER_CONSTRAINT_BLOCK
\end{verbatim}

\subsubsection{Epoch Relative to Phase Boundary}

A maneuver may be defined relative to the left or right boundary of the phase, in units of days.

\begin{verbatim}
BEGIN_MANEUVER_CONSTRAINT_BLOCK
p0b0_epoch_lboundary_14.0_10000.0
p0b0_epoch_rboundary_60.0_10000.0
END_MANEUVER_CONSTRAINT_BLOCK
\end{verbatim}

\subsubsection{Epoch Relative to Previous or Next Event}

A maneuver may be defined relative to the previous or next event in the phase (\textit{i.e.} another maneuver or a boundary), in units of days.

\begin{verbatim}
BEGIN_MANEUVER_CONSTRAINT_BLOCK
p0b0_epoch_next_14.0_10000.0
p0b0_epoch_previous_60.0_10000.0
END_MANEUVER_CONSTRAINT_BLOCK
\end{verbatim}


\subsection{Maneuver Magnitude}
\label{subsec:MGAnDSMs_constraintMagnitude}

The user may constrain the magnitude of any MGAnDSMs maneuver, in units of km/s.

\begin{verbatim}
BEGIN_MANEUVER_CONSTRAINT_BLOCK
p0b0_magnitude_0.0_0.01
END_MANEUVER_CONSTRAINT_BLOCK
\end{verbatim}

\subsection{Maneuver Thruster Set}
\label{subsec:MGAnDSMs_constrainThrusterSet}

The user may direct a particular MGAnDSMs maneuver to be either ``monoprop'' or ``biprop.'' EMTG will then use the $I_{sp}$ specified for monoprop or biprop in the spacecraft file.

\begin{verbatim}
BEGIN_MANEUVER_CONSTRAINT_BLOCK
p0b0_monoprop
END_MANEUVER_CONSTRAINT_BLOCK
\end{verbatim}


\section{Parallel Shooting Maneuver Constraints}
\label{subsec:PS_maneuver_constraints}

The parallel shooting transcriptions, \ac{PSFB} and \ac{PSBI} transcriptions decompose a low-thrust phase into \textit{n} subphases, or `steps.'' Each individual step may have up to \textit{m} maneuvers, or ``substeps,'' each of which has its own \ac{NLP} control parameters. As in \ac{MGAnDSMs}, constraints may be placed in the maneuver constraint block. \ac{PSFB} and \ac{PSBI} are designed such that constraints are placed at the step level, not at the substep level.

\subsection{Parallel Shooting Duty Cycle}
\label{subsec:PS_duty_cycle}

The user may directly constrain the duty cycle in each parallel shooting step as a floating point variable in [0.0, 1.0]. If EMTG is set to ``averaged'' duty cycle mode, this duty cycle will be applied as a multiplier to both thrust and mass flow rate. If EMTG is set to ``high-fidelity'' duty cycle mode, then the spacecraft will thrust for that percentage of the step length and then insert a forced coast until the end of the step.

\begin{verbatim}
BEGIN_MANEUVER_CONSTRAINT_BLOCK
p0b0_dutycycle_0.9
END_MANEUVER_CONSTRAINT_BLOCK
\end{verbatim}

\subsection{Parallel Shooting body-probe-thrust angle}
\label{subsec:PS_BPT}

The user may constrain the \ac{BPT} angle for any \ac{PSFB} or \ac{PSBI} phase of the CAESAR trajectory. This constraint applies to an entire phase rather than a particular step of that phase. For convergence reasons, the constraint is on the cosine of the BPT angle rather than the BPT angle itself.

The upper and lower bounds on $\cos \left(\theta_{BPT}\right)$ are expressed as functions of $r$, the distance between the spacecraft and a reference body of the user's choice. This formulation allows for BPT angle constraints that become stricter as the spacecraft flies closer to the reference body and less strict as the spacecraft flies away from the reference body. The upper and lower bounds are fit using the \ac{4PL} curve as per Equations \ref{eq:BPT_angle_bounds}.

\begin{equation}
	\label{eq:BPT_angle_bounds}
	\begin{aligned}
		L\left(r\right) \leq \cos \left(\theta_{BPT}\right) \leq U\left(r\right)\\
		U\left(r\right) = \alpha_{Ud} + \frac{\alpha_{Ua} - \alpha_{Ud}}{1.0 + \left(\frac{r}{\alpha_{Uc}}\right)^{\alpha_{Ub}}}\\
		L\left(r\right) = \alpha_{Ld} + \frac{\alpha_{La} - \alpha_{Ld}}{1.0 + \left(\frac{r}{\alpha_{Lc}}\right)^{\alpha_{Lb}}}
	\end{aligned}
\end{equation}

When specifying the constraint, the user must provide four coefficients each for the \ac{4PL} curves associated with the maximum and minimum values for $\cos \left(\theta_{BPT}\right)$, as well as the identity of the reference body. The reference body may be encoded as ``cb'' for the central body, or with an integer code describing the reference body's position in the Universe file. Note that the blow code block needs to be one line in the .emtgopt file.

Alternatively, the user may specify the constraint with a constant lower and upper bound. The bounds are given in degrees.

\begin{verbatim}
BEGIN_MANEUVER_CONSTRAINT_BLOCK
p0_bpt_cb_0.1879254,24.6087500,0.9480499,0.8878154...
           _1.6239600,14.9903500,0.8006161,-0.8878154
p0_bpt_cb_27.4_152.6
END_MANEUVER_CONSTRAINT_BLOCK
\end{verbatim}


\section{Phase Distance Constraint}
\label{sec:PhaseDistanceConstraint}

The user may constrain the distance between the spacecraft and any other body in the universe file. The body may be expressed either as ``cb'' for the central body, or with an integer code describing the reference body's position in the Universe file. The lower and upper bounds on the distance constraint may be specified in LU, AU, or km.

At the moment, this constraint works in \ac{MGALT}, \ac{FBLT}, \ac{PSBI}, and \ac{PSFB}. \ac{MGAnDSMs} does not currently have a phase distance constraint.

Each Journey block in the .emtgopt file has its own phase distance constraint section.

\begin{verbatim}
BEGIN_PHASE_DISTANCE_CONSTRAINT_BLOCK
p0_4_0.9au_10.0au
END_PHASE_DISTANCE_CONSTRAINT_BLOCK
\end{verbatim}


%\bibliographystyle{AAS_publication}
%\bibliography{EMTGbib_Jacob_June_2014}



\end{document}
