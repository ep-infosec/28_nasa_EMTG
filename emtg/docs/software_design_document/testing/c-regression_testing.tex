\chapter{Regression Testbed}
\label{chap:regression_testing}

\hl{Alec and Sean}

The \texttt{Regression Testbed} for \ac{EMTG} is, by default, located in the ``testatron/'' folder in the \ac{EMTG} installation. All test cases are located in subfolders within the ``testatron/tests/'' directory. [SEAN, MAYBE SUMMARIZE TESTS HERE?] Each test case includes an \ac{EMTG} options (.emtgopt) file for the \texttt{Regression Testbed} to rerun as well as a baseline output (.emtg) file to use as a truth value for test runs. All required universe and ephemeris files are located in the ``testatron/universe/'' folder. All required launch vehicle (.emtg\_launchvehicleopt), power system (.emtg\_powersystemsopt), propulsion system (.emtg\_propulsionsystemopt), and spacecraft (.emtg\_spacecraftopt) options files for the test cases are located in the ``testatron/HardwareModels/'' folder along with all throttle (.Throttle), throttle table (.ThrottleTable), and throttle table output (.ThrottleTableOUTPUT) files. The \texttt{Regression Testbed} uses \texttt{Testatron} (outlined in \ref{testatron}) to run and compare \ac{EMTG} cases. \texttt{Testatron} uses the \texttt{Comparatron} or \texttt{Comparatron\_NoCoast} methods (outlined in \ref{comparatronmethod}) to compare Mission, Journey, and Mission Event attributes between the new \ac{EMTG} run created by \texttt{Testatron} and the baseline output provided in the ``testatron/tests/'' folder for each test case.

\section{Testatron}
\label{testatron}

\texttt{Testatron} is the driver for the \ac{EMTG} regression testing system and is located in the ``testatron.py'' python script within the ``testatron/'' folder. \texttt{Testatron} searches through subdirectories in the ``testatron/tests/'' folder for \ac{EMTG} options files to be run and compared---all subfolders appearing in the ``testatron/tests/'' folder can be used or, alternatively, a specific set of test subfolders or cases can be provided as a list in the \texttt{Testatron} script. For each \ac{EMTG} options file found, the driver will override the output path to be a time stamped output folder in your current working directory. It also makes sure that background mode is on and that the universe and HardwareModels paths are set to those in the ``testatron'' folder. \texttt{Testatron} then saves and runs the \ac{EMTG} options file and puts the newly-generated output through a comparator (outlined in \ref{comparatronmethod}), to be validated against a baseline case.

There are multiple ways to run \texttt{Testatron}, all of which can be specified by the ``run\_type'', ``test\_cases'', and ``skip\_coasts'' arguments at the top of the script. The ``run\_type'' argument has four options:
\begin{itemize}[label=$\bullet$]
	\item $all$: Runs all test cases that appear in any subfolder in the tests directory
	\item $folders$: Runs all test cases in user-specified subfolders in the tests directory
	\item $cases$: Runs user-specified test cases
	\item $failed$: Runs only the cases that failed in a user-specified previous \texttt{Testatron} run or runs
\end{itemize}
For the $folders$, $cases$, and $failed$ options, the user must provide additional information into the ``test\_cases'' argument in order for \texttt{Testatron} to know what to run. For the $folders$ option, a list of subfolders within the ``testatron/tests/'' directory must be provided. For the $cases$ option, a list of \ac{EMTG} test case names (including full file path) must be provided. For the $failed$ option, the full file path to the \texttt{Testatron} output folder (or folders) must be provided. If \texttt{Testatron} is being run using the $all$ option, then the ``test\_cases'' argument can be ignored. Lastly, the ``skip\_coasts'' argument controls which version of \texttt{Comparatron} will be used. If ``skip\_coasts'' is set to False, the standard \texttt{Comparatron} method will be used and if ``skip\_coasts'' is set to a boolean value of True, then the \texttt{Comparatron\_NoCoast} method will be used.

As all the test cases are run, two comma-separated value files are written: ``test\_results.csv'' and ``failed\_tests.csv''. The ``test\_results.csv'' file is an overall summary file that provides a time stamp for the beginning and end of the tests as well as the successful/failed status for each individual test. The ``failed\_tests.csv'' file will only be written to if one of the tests fails in the \texttt{Comparatron}. This CSV file provides the name, values, error, and tolerance for every Mission, Journey, or Mission Event attribute that failed. 

\section{Comparatron and Comparatron\_NoCoast Methods}
\label{comparatronmethod}

\texttt{Comparatron} is a method of the Mission class (located in ``PyEMTG/Mission.py'') which compares all \ac{EMTG} output attributes against those of a baseline case. Strings are compared directly and numeric values are checked against a default tolerance of 1e-15. Alternative tolerance values for any attribute can be provided as a dictionary into the function. If all values are in agreement, \texttt{Comparatron} returns a boolean value True. If any values are not within the tolerance between the two cases, then a value of ``False" is returned along with a dataframe summarizing all inconsistencies.

The syntax for calling \texttt{Comparatron} is: ``$pass\_test=myMission.Comparatron(path\_to\_baseline,\linebreak csv\_file\_name=None,full\_output=False,tolerance\_dict=\{\},default\_tolerance=$1e-15$)$.'' Note that this is Python syntax. Any argument with an ``='' sign denotes the default value that will be used if the argument is not passed in. The function arguments are as follows:
\begin{itemize}[label=$\bullet$]
	\item $path\_to\_baseline$: The full file path to the \ac{EMTG} output file to compare against. This is the only required argument.
	\item $csv\_file\_name$: The name of the csv output file that is written if there are any discrepancies to report. If no file name is provided then the mission name will be used.
	\item $full\_output$: Dictates what is written to the csv output file. If False, only values that do not meet the tolerance will be written to the output. If True, than any values that are not in exact agreement (regardless of whether they are within the tolerance) will be written. 
	\item $tolerance\_dict$: Overrides the default tolerance for specific mission, journey, or mission event attributes. The argument takes a dictionary with the attribute as the key and the new tolerance as the value (i.e. \{`total\_statistical\_deltav':1e-6,`Declination':1e-8\}).
	\item $default\_tolerance$: Provides a default tolerance value for all attributes that do not appear in the tolerance dictionary. A value of 1e-15 will be used unless the user provides a new default value here.
\end{itemize}

When running \texttt{Comparatron}, the function will first check to make sure that both cases have the same number of journeys and mission events. If there are discrepancies in either, the function will not return the usual True/False boolean but will instead return a string saying ``Journey Mismatch" or ``Journey \# Mission Events Mismatch" and will stop immediately. If this check is passed, \texttt{Comparatron} will then check every Mission, Journey, and MissionEvent class attribute across cases and record any names that do not agree or any attributes that appear only in one---these will be written to the csv output file. The attributes for each class include both functions within the class as well as alphanumeric values (i.e. dates, the central body, and $\Delta v$). All alphanumeric-valued attributes are parsed into a temporary dataframe for the overall mission and for each journey and mission event. These temporary dataframes are split in two, one for strings and one for numerics. The strings are compared directly while the corresponding numeric dataframes between the two \ac{EMTG} cases are subtracted and checked against the tolerance. All values that do not completely agree are stored into a final comparison dataframe, which gets written to a csv output file if discrepancies are found.

\texttt{Comparatron\_NoCoast} is also a method of the Mission class and is identical to \texttt{Comparatron} in almost every way. The difference between \texttt{Comparatron\_NoCoast} and \texttt{Comparatron} is that \texttt{Comparatron\_NoCoast} ignores all Mission Events with a ``coast'' event type. This provides a faster comparator option, but no longer compares every line between two \ac{EMTG} output files since some Mission Events are being skipped.